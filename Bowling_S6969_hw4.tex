\documentclass[11pt]{article}

%%%%%%%%%%%%%%%%%%%
% Page Layout
%%%%%%%%%%%%%%%%%%%

\setlength{\paperwidth}{8.5in} \setlength{\paperheight}{11in}
\setlength{\marginparwidth}{0in} \setlength{\marginparsep}{0in}
\setlength{\oddsidemargin}{0in} \setlength{\evensidemargin}{0in}
\setlength{\textwidth}{6.5in} \setlength{\topmargin}{-0.5in}
\setlength{\textheight}{9in}

%%%%%%%%%%%%%%%%%%%%%%%%%%%%%%%%%%%
% Include Packages and Style Files
%%%%%%%%%%%%%%%%%%%%%%%%%%%%%%%%%%%

\usepackage[english]{babel}
\usepackage{amsmath,amssymb,amsthm}
\usepackage{enumerate}
%\usepackage{xcolor}
\usepackage[useregional]{datetime2}
\usepackage[pdftex]{graphicx,color}
\usepackage{listings}
\usepackage{hyperref}
\hypersetup{
    colorlinks=true,
    linkcolor=blue,
    filecolor=magenta,
    urlcolor=cyan,
}

%%%%%%%%%%%%%%%%%%%%%%%%%%%%%%
% Define theorem environments
%%%%%%%%%%%%%%%%%%%%%%%%%%%%%%

\newtheorem{theorem}{Theorem}[section]
\newtheorem{proposition}[theorem]{Proposition}
\newtheorem{lemma}[theorem]{Lemma}
\newtheorem{corollary}[theorem]{Corollary}
\newtheorem{claim}[theorem]{Claim}
\newtheorem{question}[theorem]{Question}
\newtheorem{conjecture}[theorem]{Conjecture}

\theoremstyle{definition}
\newtheorem{definition}[theorem]{Definition}
\newtheorem{example}[theorem]{Example}
\newtheorem*{remark}{Remark}

%%%%%%%%%%%%%%%%%%%%%%
% Define new commands
%%%%%%%%%%%%%%%%%%%%%%

\newcommand{\R}{\mathbb{R}}


\newcommand{\E}{\mathbb{E}}
\renewcommand{\P}{\mathbb{P}}
\newcommand{\Var}{\operatorname{Var}}
\newcommand{\1}[1]{\mathbf{1} \left \{ #1 \right \}}
\newcommand{\Range}{\operatorname{Range}}

%%%%%%%%%%%%%%%%%%%%%%

\begin{document}

\title{Practical Data Science \\ Assignment 4}
\author{Magon Bowling}
\date{February 20, 2021}

\maketitle

\section{Section: Concepts}
\begin{enumerate}
    \item Why do we want to import packages? \\
    Packages are a collection of functions and data sets developed by the programming community. They make it possible to access data sets, use specific features and functions, and code according to specified desires.
    \item What is the use of the “as” code when importing Python packages? \\
    When you import a package in Python, say pandas or numpy, the “as” command is used to reference the package in code later. This is a shorter notation that links the code to specific functions and programming.
    \item How do we save output generated by Python code? \\
    Set the output “$=$” to a variable name.
    \item How do we save output generated by R code? \\
    Store the output as a variable using “$<-$”.
    \item Why would we want to save output? \\
    To use it later in any calculations.
    \item How do we get a data set into Python? \\
    We import the data first and then perform a “read.csv” command.
    \item Why is it important to specify if our data set as column headings or not? \\
    Rows and columns are listed in a specific order, rows first. Having a specific list of the columns is necessary to reference and use throughout a code.
    \item What are the two ways we can get a data set into R? \\
    Data sets can be manually uploaded into R and then use read.csv of the file stored as a data set. Another way to get a data set into R is to read.csv the pathway to the data set, such as a url.
\end{enumerate}


\section{Section: Working with the Data}
\begin{enumerate}\addtocounter{enumi}{8}
    \item Download the program and open the compiler (both R and Python). What is contained in the bottom-right window? The left(for Python) or top-left(for R)? \\
    In both R and Python we have various windows.  There is a console to show the results of your work, there is the coding window to write and save code, and there is other window sections for files, help support, and more.
    \item Type a comment stating that you are working on Chapter 2 exercises. \\
    In both R and Python, comments are made with "\#" before the text like so:
    \[\includegraphics[height=1cm]{hw4_images/PDS_4.4.PNG}\]
    \[\includegraphics[height=1cm]{hw4_images/PDS_4.3.PNG}\]
    \item Locate the “Run” button and note whether there is a keyboard shortcut. \\
    In both R and Python programming, I use the shortcut of "ctrl+enter" to run my code.  As long as the cursor is on the line of code I want to run, then when I select "ctrl+enter" the line will run.  If I want to run more lines of code, then I select all the lines I want to run and then hit "ctrl+enter."
    \item Execute the comment from the previous exercise. What is the output? Explain your answer. \\
    When you execute a comment there is no output.  The program recognizes the text as just that and it will not run any code.
    \item Import the following packages:
    \begin{enumerate}[(1)]
        \item For Python, import the \textit{pandas} and \textit{numpy}. Rename the \textit{pandas} package “pd” and rename the \textit{numpy} package “np”.
        \[\includegraphics[height=1.25cm]{hw4_images/PDS_4.1.PNG}\]
        \item For R, import the \textit{ggplot2}. Make sure you both install and open the package.
        \[\includegraphics[height=2cm]{hw4_images/PDS_4.2.PNG}\]
    \end{enumerate}
    \item Import the "bank\_marketing\_training" data set and name it "bank\_train."
    \[\includegraphics[height=1.5cm]{hw4_images/PDS_4.6.PNG}\]
    \[\includegraphics[height=1.25cm]{hw4_images/PDS_4.5.PNG}\]
    \item Create a contingency table of the variables \textit{response} and \textit{previous\_outcome} from the "bank\_train" data set. Do not save the output from the code.
    \[\includegraphics[height=1cm]{hw4_images/PDS_4.9.PNG}\]
    \[\includegraphics[height=1cm]{hw4_images/PDS_4.8.PNG}\]
    \item Rerun the code from the previous exercise, this time saving the output as "crosstab\_01" (for Python code) or "t1" (for R code).
    \[\includegraphics[height=1.5cm]{hw4_images/PDS_4.10.PNG}\]
    \[\includegraphics[height=1.25cm]{hw4_images/PDS_4.7.PNG}\]
    \item After saving the output in the previous exercise, display the output using the name of the saved output.
    \[\includegraphics[height=3.3cm]{hw4_images/PDS_4.11a.PNG}\]
    \[\includegraphics[height=1 cm]{hw4_images/PDS_4.11b.PNG}\]
    \item Save the contingency table under a different name. This time, use your last name and favorite number as the name; for example \textit{thomas4}.
    \[\includegraphics[height=2cm]{hw4_images/PDS_4.12b.PNG}\]
    \[\includegraphics[height=2.25cm]{hw4_images/PDS_4.12a.PNG}\]
    \item Save the first nine records of the "bank\_train" data set as their own data frame.
    \[\includegraphics[height=2.2cm]{hw4_images/PDS_4.13b.PNG}\]
    \[\includegraphics[height=2cm]{hw4_images/PDS_4.13a.PNG}\]
    \item Save the \textit{age} and \textit{marital} records of the "bank\_train" data set as their own data frame.
    \[\includegraphics[height=2.5cm]{hw4_images/PDS_4.14a.PNG}\]
    \[\includegraphics[height=1.8cm]{hw4_images/PDS_4.14b.PNG}\]
    \item Save the first three records of the \textit{age} and \textit{marital} variables as their own data frame.
    \[\includegraphics[height=5cm]{hw4_images/PDS_4.15a.PNG}\]
    \[\includegraphics[height=1.8cm]{hw4_images/PDS_4.15b.PNG}\]
\end{enumerate}


\section{Section: Hands-On Analysis}
\begin{enumerate}\addtocounter{enumi}{21}
    \item Import the "adult\_ch3\_training" data set using the “Heading: Yes” setting. Rename the data set "adult" once it is imported.
    \[\includegraphics[height=1.5cm]{hw4_images/PDS_4.22r.PNG}\]
    \[\includegraphics[height=3cm]{hw4_images/PDS_4.22p.PNG}\]
    \item Write a comment explaining the change in the data set name.
    \[\includegraphics[height=.75cm]{hw4_images/PDS_4.23r.PNG}\]
    \item Import the following packages:
    \begin{enumerate}[(1)]
        \item For Python, import the \textit{DecisionTreeClassifier} command from the tree package.
        \item For R, import the \textit{rpart}. Make sure you both install and open the package.
        \[\includegraphics[height=1.5cm]{hw4_images/PDS_4.24.2.PNG}\]
    \end{enumerate}
    \item Create a contingency table of \textit{sex} and \textit{workclass} and save the output as "table01".
    \[\includegraphics[height=1cm]{hw4_images/PDS_4.25r.PNG}\]
    \[\includegraphics[height=1.7cm]{hw4_images/PDS_4.25p.PNG}\]
    \item Create a contingency table of \textit{sex} and \textit{marital.status}. Save the output as "table02".
    \[\includegraphics[height=1.2cm]{hw4_images/PDS_4.26r.PNG}\]
    \[\includegraphics[height=1.5cm]{hw4_images/PDS_4.26p.PNG}\]
    \item Display the \textit{sex} and \textit{workclass} values of the person in the first record. What cell of "table01" do they belong to? How many other records in the data set have the same \textit{sex} and \textit{workclass} values?
    \[\includegraphics[height=1.7cm]{hw4_images/PDS_4.27r.PNG}\]
    \[\includegraphics[height=6cm]{hw4_images/PDS_4.27p.PNG}\]
    \item Display the \textit{sex} and \textit{marital.status} values of the people in records 6-10. Which cells of "table02" do they belong to? How many other records in the data set have the same combinations of \textit{sex} and \textit{marital.status} values?
    \[\includegraphics[height=2cm]{hw4_images/PDS_4.28r.PNG}\]
    \[\includegraphics[height=7cm]{hw4_images/PDS_4.28p.PNG}\]
    \item Create a new data set that has only records whose \textit{marital.status} is “Married-civ-spouse” and name the data set "adultMarried".
    \[\includegraphics[height=1cm]{hw4_images/PDS_4.29r.PNG}\]
    \[\includegraphics[height=1.6cm]{hw4_images/PDS_4.29p.PNG}\]
    \item Recreate the contingency table of \textit{sex} and \textit{workclass} using the "adultMarried" data set. What differences do you notice between the sexes?
    \[\includegraphics[height=1.2cm]{hw4_images/PDS_4.30r.PNG}\]
    \[\includegraphics[height=4.5cm]{hw4_images/PDS_4.30p.PNG}\]
    \item Create a new data set that has only records whose \text{age} value is greater than 40. Name the new data set "adultOver40".
    \[\includegraphics[height=1cm]{hw4_images/PDS_4.31r.PNG}\]
    \[\includegraphics[height=4.5cm]{hw4_images/PDS_4.31p.PNG}\]
    \item Recreate a contingency table of \textit{sex} and \textit{marital status} using the "adultOver40" data set. What differences do you notice?
    \[\includegraphics[height=1.4cm]{hw4_images/PDS_4.32r.PNG}\]
    \[\includegraphics[height=4.5cm]{hw4_images/PDS_4.32p.PNG}\]
\end{enumerate}


\section{Section: Concepts}
\begin{enumerate}\addtocounter{enumi}{32}
    \item Describe two reasons why it might be a good idea to add an index field to the data set. \\
    An index field is valuable in a data set because it assigns a number to a row or column making it easier to call specific parts of the data.  This allows for subsets to be created as well using an index field.
    \item Explain why the field \textit{days\_since\_previous} is essentially useless until we handle the $999$ code. \\
    The $999$ code is a way to communicate that there is no value or data collected.  These values are outliers in the data set.  It is essential to convert these outliers useless values to "NA" in R or "NaN" in Python for them to not impact the data and visualizations.
    \item Why was it important to re-express \textit{education} as a numeric field? \\
    Numeric fields can be calculated on, whereas, character fields cannot have functions coded with them.  Changing \textit{education} to numeric allows us to use functions within that field.
    \item Suppose a data value has a \textit{z-value} of 1.  How may we interpret this value? \\
    z-values are used to help identify outliers with a scaled value.  A 1 would be interpreted as a value that is not an outlier.
    \item What is the rough rule of thumb for identifying outliers using z-values? \\
    If z-values are $< -3$ or $> 3$ then they are considered outliers.
    \item Should outliers be automatically removed or changed?  Why or why not? \\
    I personally do not think that outliers should be removed!  I do think that, depending on the analytics that need to be made, outliers can be changed.  However, outliers have a reason and a purpose and it is important to know what they are and why and then note it in the explanation of the data.
    \item What should we do with outliers we have identified? \\
    We can change the value to "NA" in R or "NaN" in Python like explained, or they can be ignored through code, or you can delete them... Not the best idea.
\end{enumerate}


\section{Section: Working with the Data}
\begin{enumerate}\addtocounter{enumi}{39}
    \item Derive an index field and add it to the data set.
    \[\includegraphics[height=1.6cm]{hw4_images/PDS_4.40r.PNG}\]
    \[\includegraphics[height=7cm]{hw4_images/PDS_4.40p.PNG}\]
    \item For the \textit{days\_since\_previous} field, change the field value $999$ to the appropriate code for missing values, as shown in class.
    \[\includegraphics[height=1.2cm]{hw4_images/PDS_4.41r.PNG}\]
    \[\includegraphics[height=1.3cm]{hw4_images/PDS_4.41p.PNG}\]
    \item For the \textit{education} field, re-express the field values as the numeric values shown in class.
    \[\includegraphics[height=4cm]{hw4_images/PDS_4.42r.PNG}\]
    \[\includegraphics[height=6.4cm]{hw4_images/PDS_4.42p.PNG}\]
    \item Standardize the field \textit{age}.  Print out a list of the first $10$ records, including the variables \textit{age} and \textit{age\_z}.
    \[\includegraphics[height=1.2cm]{hw4_images/PDS_4.43r.PNG}\]
    \[\includegraphics[height=2cm]{hw4_images/PDS_4.43p.PNG}\]
    \item Obtain a listing of all records that are outliers according to the field \textit{age\_z}.  Print out a listing of the $10$ largest \textit{age\_z}.
    \[\includegraphics[height=2.2cm]{hw4_images/PDS_4.44r.PNG}\]
    \[\includegraphics[height=6cm]{hw4_images/PDS_4.44p.PNG}\]
    \item For the \textit{job} field, combine the jobs with less than $5\%$ of the records into a field called \textit{other}.
    \item Rename the \textit{default} predictor to \textit{credit\_default}.
    \[\includegraphics[height=1.5cm]{hw4_images/PDS_4.46r.PNG}\]
    \[\includegraphics[height=4cm]{hw4_images/PDS_4.46p.PNG}\]
    \item For the variable \textit{month}, change the field values to $1-12$, but keep the variable as categorical.
    \[\includegraphics[height=4.5cm]{hw4_images/PDS_4.47r.PNG}\]
    \[\includegraphics[height=4cm]{hw4_images/PDS_4.47p.PNG}\]
    \item Do the following for the \textit{duration}
    \begin{enumerate}[(1)]
        \item Standardize the variable.
        \item Identify how many outliers there are and identify the most extreme outlier.
    \end{enumerate}
    \[\includegraphics[height=2.25cm]{hw4_images/PDS_4.48r.PNG}\]
    \[\includegraphics[height=3cm]{hw4_images/PDS_4.48p.PNG}\]
    \item Do the following for the \textit{campaign}
    \begin{enumerate}[(1)]
        \item Standardize the variable.
        \item Identify how many outliers there are and identify the most extreme outlier.
        \end{enumerate}
        \[\includegraphics[height=2.25cm]{hw4_images/PDS_4.49r.PNG}\]
        \[\includegraphics[height=3cm]{hw4_images/PDS_4.49p.PNG}\]
\end{enumerate}


\section{Section: Hands-On Analysis}
For questions $50-53$, work with the "Nutrition\_subset" data set.  The data set contains the weight in grams along with the amount of saturated fat and the amount of cholesterol for a set of $961$ foods.  Use either Python or R to solve each problem.
\begin{enumerate}\addtocounter{enumi}{49}
    \item The elements in the data set are food items of various sizes, ranging from a teaspoon of cinnamon to an entire carrot cake.
    \begin{enumerate}[(1)]
        \item Sort the data set by the saturated fat (\textit{saturated\_fat}) and produce a listing of the five food items highest in saturated fat.
        \[\includegraphics[height=5cm]{hw4_images/PDS_4.50.1p.PNG}\]
        \item Comment on the validity of comparing food items of different sizes.
        \[\includegraphics[height=1.2cm]{hw4_images/PDS_4.50.2p.PNG}\]
    \end{enumerate}
    \item Derive a new variable, \textit{saturated\_fat\_per\_gram}, by dividing the amount of saturated fat by the weight in grams.
     \[\includegraphics[height=5cm]{hw4_images/PDS_4.51p.PNG}\]
    \begin{enumerate}[(1)]
        \item Sort the data set by \textit{saturated\_fat\_per\_gram}, by dividing the amount of saturated fat by the weight in grams.
        \item Which food has the most saturated fat per gram?
        \[\includegraphics[height=3.5cm]{hw4_images/PDS_4.51.1p.PNG}\]
    \end{enumerate}
    \item Derive a new variable,
    \begin{enumerate}[(1)]
        \item Sort the data set by \textit{cholesterol\_per\_gram} and produce a listing of the five food items highest in cholesterol fat per gram.
        \[\includegraphics[height=4.6cm]{hw4_images/PDS_4.52p.PNG}\]
        \item Which food has the most cholesterol fat per gram?
        \[\includegraphics[height=1.7cm]{hw4_images/PDS_4.52.2p.PNG}\]
    \end{enumerate}
    \item Standardize the field \textit{saturated\_fat\_per\_gram}.  Produce a listing of all the food items that are outliers at the high end of the scale.  How many food items are outliers at the low end of the scale?
    \[\includegraphics[height=10cm]{hw4_images/PDS_4.53p.PNG}\]
    \[\includegraphics[height=2cm]{hw4_images/PDS_4.53_p.PNG}\]
\end{enumerate}
For questions $54-58$, work with the \textit{adult\_ch3\_training} data set. The response is whether income exceeds $\$50,000$.
\begin{enumerate}\addtocounter{enumi}{53}
    \item Add a record index field to the data set.
    \[\includegraphics[height=3cm]{hw4_images/PDS_4.54p.PNG}\]
    \item Determine whether any outliers exist for the \textit{education}.
    \[\includegraphics[height=2.8cm]{hw4_images/PDS_4.55p.PNG}\]
    \item Do the following for the \textit{age}.
    \begin{enumerate}[(1)]
        \item Standardize the variable.
        \item Identify how many outliers there are and identify the most extreme outlier.
    \end{enumerate}
    \[\includegraphics[height=3cm]{hw4_images/PDS_4.56p.PNG}\]
    \item Derive a flag for \textit{capital-gain}, called \textit{capital-gain-flag}, which equals $0$ for capital gain, and $1$ otherwise.
    \[\includegraphics[height=7cm]{hw4_images/PDS_4.57p.PNG}\]
    \item Age anomaly? Select only records with age at least $80$.  Construct a histogram of age.  Explain what you see in one sentence and why it is like that in another sentence.
    \[\includegraphics[height=9cm]{hw4_images/PDS_4.58p.PNG}\]
    We have peak of records with age 89 compared to frequencies half the size or less from 80-88, with no records older than 90.  This may be that anyone 89 and older are grouped together into one age category.
\end{enumerate}


\section{Section: Tableau}
\begin{enumerate}\addtocounter{enumi}{58}
    \item Create an interactive Tableau dashboard for the SLC restaurants with an embedded Google map. \\
    Visit \href{https://public.tableau.com/profile/magon7952#!/vizhome/Bowling\_PDS\_Assignment4_Restaurants/Restaurants?publish=yes}{Magon's Tableau Public} for this portion.
\end{enumerate}


\end{document}
