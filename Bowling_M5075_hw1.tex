\documentclass[11pt]{article}

%%%%%%%%%%%%%%%%%%%
% Page Layout
%%%%%%%%%%%%%%%%%%%

\setlength{\paperwidth}{8.5in} \setlength{\paperheight}{11in}
\setlength{\marginparwidth}{0in} \setlength{\marginparsep}{0in}
\setlength{\oddsidemargin}{0in} \setlength{\evensidemargin}{0in}
\setlength{\textwidth}{6.5in} \setlength{\topmargin}{-0.5in}
\setlength{\textheight}{9in}

%%%%%%%%%%%%%%%%%%%%%%%%%%%%%%%%%%%
% Include Packages and Style Files
%%%%%%%%%%%%%%%%%%%%%%%%%%%%%%%%%%%

\usepackage[english]{babel}
\usepackage{amsmath,amssymb,amsthm}
\usepackage{enumerate}
\usepackage{xcolor}
\usepackage[useregional]{datetime2}
\usepackage[pdftex]{graphicx,color}
\usepackage{graphicx}
\usepackage{multicol}
\setlength{\columnsep}{1.5cm}

%%%%%%%%%%%%%%%%%%%%%%%%%%%%%%
% Define theorem environments
%%%%%%%%%%%%%%%%%%%%%%%%%%%%%%

\newtheorem{theorem}{Theorem}[section]
\newtheorem{proposition}[theorem]{Proposition}
\newtheorem{lemma}[theorem]{Lemma}
\newtheorem{corollary}[theorem]{Corollary}
\newtheorem{claim}[theorem]{Claim}
\newtheorem{question}[theorem]{Question}
\newtheorem{conjecture}[theorem]{Conjecture}

\theoremstyle{definition}
\newtheorem{definition}[theorem]{Definition}
\newtheorem{example}[theorem]{Example}
\newtheorem*{remark}{Remark}

%%%%%%%%%%%%%%%%%%%%%%
% Define new commands
%%%%%%%%%%%%%%%%%%%%%%

\newcommand{\R}{\mathbb{R}}


\newcommand{\E}{\mathbb{E}}
\renewcommand{\P}{\mathbb{P}}
\newcommand{\Var}{\operatorname{Var}}
\newcommand{\1}[1]{\mathbf{1} \left \{ #1 \right \}}
\newcommand{\Range}{\operatorname{Range}}

%%%%%%%%%%%%%%%%%%%%%%

\begin{document}

\title{Time Series Analysis \\ Homework 1}
\date{Due: Friday, January 29th}
\author{Magon Bowling \\ Math 5075}

\maketitle

\section{{\color{red} \textbf{Problem 1}}}

\item Let $X$ and $Y$ be independent and identically distributed exponential random variables with \(EX = EY = 1\).  Compute the distribution function of \(Z = X + Y \).
\\
\item We are given that \(EX = EY = 1\), and we know that $EX$ of an exponential random variable is \(\frac{1}{\lambda}\). Thus, we have the random variables \(X \sim exp(1)\) and \(Y \sim exp(1)\).  Since $X$ and $Y$ are non-negative, $X + Y$ is also non-negative.  Let $Z = X + Y$, this means that \(f_Z (z) = 0\) if $z<0$.  If $z \geq 0$ then
\[f_Z (z) = \iint f_X (x) f_Y (y) dx dy = \int_0^z \int_0^{z-x} f_X (x) f_Y (y) dx dy = \int_0^z f_X (x) f_Y (z-x) dx\]
\[= \int_0^z e^{-x} e^{-(z-x)} dx = \int_0^z e^{-z} dx = xe^{-z}\BigZ = ze^{-z}, z>0 \]
\\
This identifies $Z$ as a Gamma(2,1).

\section{{\color{red} \textbf{Problem 3}}}

\item Let $X_1, X_2, ..., X_n$ be independent and identically distributed random variables with distribution function
\[F(x) =
\begin{cases}
0&\text{, $x < 0$} \\ 1 - \frac{1}{1+x}&\text{, $x \geq 0$.}
\end{cases}
\]
\[\includegraphics[width=6cm, height=2cm]{M5075/M5075_hw1_problem3.PNG}\]
\\ Let \(X_{n,n} = max_{1 \leq i \leq n} X_i\) and \(Y_n = X_{n,n}/n\).  Show that $Y_n$ converges in distribution and determine the limit.
\\
\begin{align*}
    \begin{split}
        F_{Y_n} (x) & = F_{\frac{X_{n,n}}{n}} (x) \\
        & = P\bigg(\frac{X_{n,n}}{n} \leq x\bigg) \\
        & = P(X_{n,n} \leq nx) \\
        & = P(max_{i \leq n} X_i \leq nx) \\
        & = P(X_1 \leq nx \cap X_2 \leq nx \cap ... \cap X_n \leq nx) \\
        & = P\Big(\cap_{i=1}^n X_i \leq nx\Big) \\
        & = \prod_{i=1}^{n} P(X_i \leq nx) \\
        & = P(X_i \leq nx)^n \\
        & = F_1 (nx)^n \\
        & = \bigg(1 - \frac{1}{1 + nx}\bigg)^n \\
        & = \bigg(\frac{1 + nx - 1}{1 + nx} \bigg)^n \\
        & = \bigg(\frac{nx}{1 + nx}\bigg)^n
   \end{split}
\end{align*}
\\
Now we need to show that $Y_n$ converges in distribution by taking the limit. \\
\[ \lim_{n\to\infty} F_{Y_n} (x) = e^{\frac{1}{x}} , x \geq 0.\]

\end{document}
