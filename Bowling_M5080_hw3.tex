\documentclass[11pt]{article}

%%%%%%%%%%%%%%%%%%%
% Page Layout
%%%%%%%%%%%%%%%%%%%

\setlength{\paperwidth}{8.5in} \setlength{\paperheight}{11in}
\setlength{\marginparwidth}{0in} \setlength{\marginparsep}{0in}
\setlength{\oddsidemargin}{0in} \setlength{\evensidemargin}{0in}
\setlength{\textwidth}{6.5in} \setlength{\topmargin}{-0.5in}
\setlength{\textheight}{9in}

%%%%%%%%%%%%%%%%%%%%%%%%%%%%%%%%%%%
% Include Packages and Style Files
%%%%%%%%%%%%%%%%%%%%%%%%%%%%%%%%%%%

\usepackage[english]{babel}
\usepackage{amsmath,amssymb,amsthm}
\usepackage{enumerate}
\usepackage{xcolor}
\usepackage[useregional]{datetime2}
\usepackage[pdftex]{graphicx,color}

%%%%%%%%%%%%%%%%%%%%%%%%%%%%%%
% Define theorem environments
%%%%%%%%%%%%%%%%%%%%%%%%%%%%%%

\newtheorem{theorem}{Theorem}[section]
\newtheorem{proposition}[theorem]{Proposition}
\newtheorem{lemma}[theorem]{Lemma}
\newtheorem{corollary}[theorem]{Corollary}
\newtheorem{claim}[theorem]{Claim}
\newtheorem{question}[theorem]{Question}
\newtheorem{conjecture}[theorem]{Conjecture}

\theoremstyle{definition}
\newtheorem{definition}[theorem]{Definition}
\newtheorem{example}[theorem]{Example}
\newtheorem*{remark}{Remark}

%%%%%%%%%%%%%%%%%%%%%%
% Define new commands
%%%%%%%%%%%%%%%%%%%%%%

\newcommand{\R}{\mathbb{R}}


\newcommand{\E}{\mathbb{E}}
\renewcommand{\P}{\mathbb{P}}
\newcommand{\Var}{\operatorname{Var}}
\newcommand{\1}[1]{\mathbf{1} \left \{ #1 \right \}}
\newcommand{\Range}{\operatorname{Range}}

%%%%%%%%%%%%%%%%%%%%%%

\begin{document}

\title{Statistical Inference I \\ Homework 3}
\date{Due: Saturday, February 13th}
\author{Magon Bowling}

\maketitle

\begin{itemize}
\item [{\color{red} \textbf{Meyer, 30.11}}] A device has two components, and the lifetimes of these components are modeled by the random variables $Y_1$ and $Y_2$. The first component cannot fail before the second component fails, and the joint distribution of $Y_1$ and $Y_2$ is determined to be
\[
f(y_1, y_2) = 2 e^{-y_1} e^{-y_2}, \1{0 < y_2 < y_1 < \infty}.
\]
The random variable $X_1 = Y_1 - Y_2$ can be interpreted as the time between failures, and $X_2 = Y_1 + Y_2$ can be interpreted as the lifetime of the device.
\begin{enumerate}[(a)]
\item Find the joint density of $X_1$ and $X_2$.
\item Find the marginal density of $X_1$ and sketch it.
\item Find the marginal density of $X_2$.
\end{enumerate}
\end{itemize}
(a) To find the joint density of $X_1$ and $X_2$ we complete the following steps: \\
\textbf{First} write relations in little variables.
\[X_1 = Y_1 - Y_2 \Rightarrow x_1 = y_1 - y_2\]
\[X_2 = Y_1 + Y_2 \Rightarrow x_2 = y_1 + y_2\]
\textbf{Second} solve for $y_1, y_2$ in terms of $x_1, x_2$.
\[x_1 = y_1 - y_2 \ (+) \ x_2 = y_1 + y_2 \Rightarrow x_1 + x_2 = 2y_1 \Rightarrow y_1 = \frac{1}{2}(x_1 + x_2)\]
\[x_1 = y_1 + y_2 \ (-) \ x_2 = y_1 + y_2 \Rightarrow x_1 - x_2 = 2y_2 \Rightarrow y_2 = \frac{1}{2}(x_2 - x_1)\]
\textbf{Third} compute the \textbf{Jacobian}.
\[\textbf{J} = \text{det}\begin{pmatrix}
\frac{\partial x_1}{\partial y_1} & \frac{\partial x_1}{\partial y_2} \\
\frac{\partial x_2}{\partial y_1} & \frac{\partial x_2}{\partial y_2} \end{pmatrix}
= \text{det}\begin{bmatrix}
\frac{1}{2} & \frac{-1}{2} \\
\frac{1}{2} & \frac{1}{2} \end{bmatrix}
= \frac{1}{2}\left(\frac{1}{2}\right) - \frac{1}{2}\left(\frac{-1}{2}\right) = \frac{1}{4} - \left(\frac{-1}{4}\right) = \frac{1}{2}\]
\textbf{Fourth} then find the formula in terms of $f_{Y_1,Y_2}$.
\begin{align*}
    f_{(X_1,X_2)} (x_1,x_2) &= f_{(Y_1,Y_2)} (y_1,y_2)|\textbf{J}| \\
    &= \left|\frac{1}{2}\right| 2 e^{-\left(\frac{1}{2}(x_1 + x_2)\right)} e^{-\left(\frac{1}{2}(x_2 - x_1)\right)}, \1{0 < \frac{1}{2}(x_2 - x_1) < \frac{1}{2}(x_1 + x_2) < \infty} \\
    &= exp\{-\frac{1}{2}x_1 - \frac{1}{2}x_2 -\frac{1}{2}x_2 + \frac{1}{2}x_1\}, \1{0 < x_2 - x_1 < x_1 + x_2 < \infty} \\ &= e^{-x_2}, \1{0 < x_1 < x_2 < \infty}
\end{align*}

(b) To find the marginal density of $X_1$ we integrate out $x_2$: \\
\begin{minipage}{0.6\linewidth}
    \begin{align*}
        f_{X_1} (x_1) &= \int_{x_1}^{\infty} e^{-x_2} dx_2 \\
        &= -e^{-x_2} \Big|_{x_1}^{\infty} \\
        &= 0 - (-e^{-x_1}) \\
        &= e^{-x_1}, x_1 > 0
    \end{align*}
\end{minipage}
\begin{minipage}{0.4\linewidth}
    \includegraphics[width=5cm]{Images/f(x_1) = e^-x_1.PNG}
\end{minipage} \\
\\
(c) To find the marginal density of $X_2$ we integrate out $x_1$: \\
\begin{minipage}{0.6\linewidth}
    \begin{align*}
        f_{X_2} (x_2) &= \int_0^{x_2} e^{-x_2} dx_1 \\
        &= e^{-x_2} x_1 \Big|_0^{x_2} \\
        &= x_2 e^{-x_2} - 0 \\
        &= x_2 e^{-x_2}, x_2 > 0
    \end{align*}
\end{minipage}
\begin{minipage}{0.4\linewidth}
    \includegraphics[width=5cm]{Images/f(x_2) = x_2 e^-x_2.PNG}
\end{minipage} \\

\begin{itemize}
\item [{\color{red} \textbf{Meyer, 30.9}}] The lifetimes $Y_1$ and $Y_2$ of two components of a device are jointly distributed as
\[
f(y_1, y_2) = \frac{1}{8} y_1 e^{-(y_1 + y_2)/2} \1{y_1 > 0, y_2 > 0}.
\]
\begin{enumerate}[(a)]
\item Find $\P(Y_1 > 1, Y_2 > 1)$.
\item Are $Y_1$ and $Y_2$ independent random variables? Explain why or why not.
\item Find the marginal density of $Y_1$.
\end{enumerate}
\end{itemize}
(a) To find $\P(Y_1 > 1, Y_2 > 1)$ we integrate on the interval $[1,\infty]$.
\begin{align*}
    \P(Y_1 > 1, Y_2 > 1) &= \int_1^{\infty} \int_1^{\infty} \frac{1}{8} y_1 e^{-(y_1 + y_2)/2} dy_1 dy_2 \\
    &= \int_1^{\infty} \int_1^{\infty} \frac{1}{8} y_1 e^{-y_1/2} e^{-y_2/2} dy_1 dy_2 \\
    &= \int_1^{\infty} \frac{1}{4} y_1 e^{-y_1/2} dy_1 \int_1^{\infty} \frac{1}{2} e^{-y_2/2} dy_2 \\
    &= \left(\frac{1}{2} y_1 e^{-y_1/2} \Big|_1^{\infty} - e^{-y_1/2} \Big|_1^{\infty}\right) \left(-e^{-y_2/2} \Big|_1^{\infty}\right) \\
    &= \left[0 - \left(\frac{-1}{2} e^{\frac{-1}{2}}\right) - \left(0 - \left(e^\frac{-1}{2}\right)\right)\right] \left(0 - \left(-e^\frac{-1}{2}\right)\right) \\
    &= \left(\frac{1}{2} e^{\frac{-1}{2}} + e^{\frac{-1}{2}}\right)\left(e^{\frac{-1}{2}}\right) \\
    &= \left(\frac{3}{2} e^{\frac{-1}{2}}\right)\left(e^{\frac{-1}{2}}\right) \\
    &= \frac{3}{2e}
\end{align*}
(b) To determine if $Y_1$ and $Y_2$ are independent, we need the product of the marginal densities to equal the joint density.  I will prove this hereafter.  We can also prove this by showing that the joint density can factor into a function of two marginal densities as indicated above in our integration.  Therefore, $Y_1$ and $Y_2$ are independent. \\
(c) To find the marginal of $Y_1$ we integrate out $y_2$:
\begin{align*}
    f_{Y_1} (y_1) &= \int_0^{\infty} \frac{1}{8} y_1 e^{-(y_1 + y_2)/2} dy_2 \\
    &= \frac{1}{4} y_1 e^{-y_1/2} \int_0^{\infty} \frac{1}{2} e^{-y_2/2} dy_2 \\
    &= \frac{1}{4} y_1 e^{-y_1/2} \left(-e^{-y_2/2} \Big|_0^{\infty} \right) \\
    &= \frac{1}{4} y_1 e^{-y_1/2} (0 - (-1)) \\
    &= \frac{1}{4} y_1 e^{-y_1/2}, \1{y_1 >0}
\end{align*}

To find the marginal of $Y_2$ we integrate out $y_1$:
\begin{align*}
    f_{Y_2} (y_2) &= \int_0^{\infty} \frac{1}{8} y_1 e^{-(y_1 + y_2)/2} dy_1 \\
    &= \frac{1}{2} e^{-y_2/2} \int_0^{\infty} \frac{1}{4} y_1 e^{-y_1/2} dy_1 \\
    &= \frac{1}{2} e^{-y_2/2} \left(\frac{1}{2} y_1 e^{-y_1/2} \Big|_0^{\infty} - e^{-y_1/2} \Big|_0^{\infty} \right) \\
    &= \frac{1}{2} e^{-y_2/2} (0 - (0 - (1))) \\
    &= \frac{1}{2} e^{-y_2/2}, \1{y_2 >0}
\end{align*}
To prove independence further,
\begin{align*}
    f_{Y_1,Y_2} (y_1, y_2) &= f_{Y_1} (y_1) f_{Y_2} (y_2) \\
    \frac{1}{8} y_1 e^{-(y_1 + y_2)/2} &= \left(\frac{1}{4} y_1 e^{-y_1/2}\right) \left(\frac{1}{2} e^{-y_2/2}\right) \\
    \frac{1}{8} y_1 e^{-(y_1 + y_2)/2} &= \frac{1}{8} y_1 e^{-y_1/2} e^{-y_2/2} \\
    \frac{1}{8} y_1 e^{-(y_1 + y_2)/2} &= \frac{1}{8} y_1 e^{-(y_1 + y_2)/2}.
\end{align*}

\end{document}
