\documentclass[11pt]{article}

%%%%%%%%%%%%%%%%%%%
% Page Layout
%%%%%%%%%%%%%%%%%%%

\setlength{\paperwidth}{8.5in} \setlength{\paperheight}{11in}
\setlength{\marginparwidth}{0in} \setlength{\marginparsep}{0in}
\setlength{\oddsidemargin}{0in} \setlength{\evensidemargin}{0in}
\setlength{\textwidth}{6.5in} \setlength{\topmargin}{-0.5in}
\setlength{\textheight}{9in}

%%%%%%%%%%%%%%%%%%%%%%%%%%%%%%%%%%%
% Include Packages and Style Files
%%%%%%%%%%%%%%%%%%%%%%%%%%%%%%%%%%%

\usepackage[english]{babel}
\usepackage{amsmath,amssymb,amsthm}
\usepackage{enumerate}
\usepackage{xcolor}
\usepackage[useregional]{datetime2}
%\usepackage[pdftex]{graphicx,color}

%%%%%%%%%%%%%%%%%%%%%%%%%%%%%%
% Define theorem environments
%%%%%%%%%%%%%%%%%%%%%%%%%%%%%%

\newtheorem{theorem}{Theorem}[section]
\newtheorem{proposition}[theorem]{Proposition}
\newtheorem{lemma}[theorem]{Lemma}
\newtheorem{corollary}[theorem]{Corollary}
\newtheorem{claim}[theorem]{Claim}
\newtheorem{question}[theorem]{Question}
\newtheorem{conjecture}[theorem]{Conjecture}

\theoremstyle{definition}
\newtheorem{definition}[theorem]{Definition}
\newtheorem{example}[theorem]{Example}
\newtheorem*{remark}{Remark}

%%%%%%%%%%%%%%%%%%%%%%
% Define new commands
%%%%%%%%%%%%%%%%%%%%%%

\newcommand{\R}{\mathbb{R}}


\newcommand{\E}{\mathbb{E}}
\renewcommand{\P}{\mathbb{P}}
\newcommand{\Var}{\operatorname{Var}}
\newcommand{\1}[1]{\mathbf{1} \left \{ #1 \right \}}
\newcommand{\Range}{\operatorname{Range}}

%%%%%%%%%%%%%%%%%%%%%%

\begin{document}

\title{Homework 1}
\date{Due: Saturday, January 30th at 11:59 PM}
\author{Magon Bowling \\ Math 5080}

\maketitle

\begin{itemize}

\item [{\color{red} \textbf{6.03}}] The measured radius $R$ of a circle has pdf $f(r) = 6r(1-r), 0 < r < 1$.
\begin{enumerate}[(a)]
\item Find the distribution of the circumference.

A circumference of a circle is \(X = 2 \pi R\).  The function \(\phi(x) = 2 \pi x \) is one-to-one and is increasing on [0,1], so it maps [0,1] onto [0,2$\pi$].  It's inverse is \(\phi^{-1}(r) = \frac{x}{2\pi}\), so
\[f_X (x) = f_X \big(\phi^{-1}(r)\big) \Big|(\phi^{-1)\prime}(r)\Big| = f_X \bigg(\frac{x}{2\pi}\bigg) \bigg|\frac{1}{2\pi}\bigg| = \frac{6x}{2\pi} \bigg(2\pi - \frac{x}{2\pi}\bigg) \bigg(\frac{1}{2\pi}\bigg)\]
\[f_X (x) = \frac{6x(2\pi - x)}{(2\pi)^3} \{0<x<2\pi\}. \]


\item Find the distribution of the area of the circle.

The area of a circle is \(Y = \pi R^2\).  The function \(\phi(y) = \pi x^2 \) is not one-to-one, so when mapping \(x \rightarrow x^2 \pi\) it maps [0,1] onto [0,$\pi$] with each point mapped twice.  The function \(\phi(y) = \pi x^2 \) has an inverse of \(\phi^{-1}(r) = \sqrt{\frac{y}{\pi}}\), so
\[f_Y (y) = f_Y \big(\phi^{-1}(r)\big)\Big|(\phi^{-1)\prime}(r)\Big| = f_Y \bigg(\sqrt{\frac{y}{\pi}}\bigg) \bigg|\frac{1}{2\sqrt{\pi y}}\bigg| = 6 \bigg(\sqrt{\frac{y}{\pi}}\bigg) \bigg(1 - \sqrt{\frac{y}{\pi}}\bigg) \bigg(\frac{1}{2\sqrt{\pi y}}\bigg)\]
\[f_Y (y) = \frac{3}{\pi} \bigg(1 - \sqrt{\frac{y}{\pi}}\bigg)  \{0<y<\pi\} \]
or
\[f_Y (y) = \frac{3(\sqrt{\pi} - \sqrt{y})}{\pi^\frac{3}{2}}  \{0<y<\pi\}. \]

\end{enumerate}

\item [{\color{red} \textbf{6.13}}] Let $X$ have pdf $f(x) = x^2/24$ for $-2 < x < 4$ and assume the pdf is zero otherwise. Find the pdf of $Y = X^2$.

The function \(\phi(x) = x^2\) is not one-to-one on [-2,4].  We discover that when mapping \(x \rightarrow x^2\) it maps [-2,4] onto [4,16].  It's inverse is \(\phi^{-1}(x) = \sqrt{y}\).  Thus we obtain,
\[f_Y (y) = f_Y \big(\phi^{-1}(x)\big)\Big|(\phi^{-1)\prime}(x)\Big| = f_Y (\sqrt{y}) \bigg|\frac{1}{\sqrt{y}}\bigg| = \bigg(\frac{(\sqrt{y})^2}{24}\bigg) \bigg(\frac{1}{2\sqrt{y}}\bigg) \]
\[f_Y (y) = \frac{\sqrt{y}}{48} \{4\leq y<16\}. \]

\end{itemize}

\end{document}
