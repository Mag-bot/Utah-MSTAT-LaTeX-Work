\documentclass[11pt]{article}

%%%%%%%%%%%%%%%%%%%
% Page Layout
%%%%%%%%%%%%%%%%%%%

\setlength{\paperwidth}{8.5in} \setlength{\paperheight}{11in}
\setlength{\marginparwidth}{0in} \setlength{\marginparsep}{0in}
\setlength{\oddsidemargin}{0in} \setlength{\evensidemargin}{0in}
\setlength{\textwidth}{6.5in} \setlength{\topmargin}{-0.5in}
\setlength{\textheight}{9in}

%%%%%%%%%%%%%%%%%%%%%%%%%%%%%%%%%%%
% Include Packages and Style Files
%%%%%%%%%%%%%%%%%%%%%%%%%%%%%%%%%%%

\usepackage[english]{babel}
\usepackage{amsmath,amssymb,amsthm}
\usepackage{enumerate}
%\usepackage{xcolor}
\usepackage[useregional]{datetime2}
\usepackage[pdftex]{graphicx,color}

%%%%%%%%%%%%%%%%%%%%%%%%%%%%%%
% Define theorem environments
%%%%%%%%%%%%%%%%%%%%%%%%%%%%%%

\newtheorem{theorem}{Theorem}[section]
\newtheorem{proposition}[theorem]{Proposition}
\newtheorem{lemma}[theorem]{Lemma}
\newtheorem{corollary}[theorem]{Corollary}
\newtheorem{claim}[theorem]{Claim}
\newtheorem{question}[theorem]{Question}
\newtheorem{conjecture}[theorem]{Conjecture}

\theoremstyle{definition}
\newtheorem{definition}[theorem]{Definition}
\newtheorem{example}[theorem]{Example}
\newtheorem*{remark}{Remark}

%%%%%%%%%%%%%%%%%%%%%%
% Define new commands
%%%%%%%%%%%%%%%%%%%%%%

\newcommand{\R}{\mathbb{R}}


\newcommand{\E}{\mathbb{E}}
\renewcommand{\P}{\mathbb{P}}
\newcommand{\Var}{\operatorname{Var}}
\newcommand{\1}[1]{\mathbf{1} \left \{ #1 \right \}}
\newcommand{\Range}{\operatorname{Range}}

%%%%%%%%%%%%%%%%%%%%%%

\begin{document}

\title{Statistical Inference I \\ Homework 5}
\date{Due: Saturday, February 27th}
\author{Magon Bowling}

\maketitle

\begin{itemize}
\item [{\color{red} \textbf{8.9}}] Supposes that $X \sim \chi^2(m), \ S = X+Y \sim \chi^2(m+n)$, and $X$ and $Y$ are independent. Is $S - X \sim \chi^2(n)$?
\end{itemize}
We know \(S = X+Y \sim \chi^2(m+n)\) and because $X$ and $Y$ are independent we have
\[M_S (t) = M_{X+Y} (t) = M_X (t) M_Y (t).\]
We can further say that because $X \sim \chi^2(m)$ and $X+Y \sim \chi^2(m+n)$ that
\[M_S (t) = (1-2t)^{\frac{-(m+n)}{2}} = (1-2t)^{\frac{-m}{2}} (1-2t)^{\frac{-n}{2}} = M_X (t) M_Y (t) = M_{X+Y} (t) = M_S (t).\]
We have shown that $Y \sim \chi^2(n)$ and we know
\[M_Y (t) = M_{X+Y-X} (t) = M_{S-X} (t),\]
therefore
\[S - X \sim \chi^2(n).\]

\begin{itemize}
\item [{\color{red} \textbf{8.14}}] If $T \sim t(\nu)$, give the distribution of $T^2$.
\end{itemize}
We know that for $T \sim t(\nu)$ we have
\[T = \frac{Z}{\sqrt{V/\nu}}\]
where $Z \sim N(0,1)$ and $V \sim \chi^2(\nu)$ is independent of $Z$.  Now if we square both sides we get
\[T^2 = \left(\frac{Z}{\sqrt{V/\nu}}\right)^2 = \frac{Z^2}{V/\nu}.\]
We also know that $Z^2 \sim \chi^2(1)$, thus we have
\[\frac{Z^2}{V/\nu} = \frac{\chi^2 (1)}{V/\nu} = \frac{\chi^2 (1)/1}{V/\nu}.\]
By definition of $F$ distribution, $T^2 \sim F(1,\nu)$.

\begin{itemize}
\item [{\color{red} \textbf{8.19}}] If $T \sim t(1)$ then show the following:
\begin{enumerate}[(a)]
\item The CDF of $T$ is $F(t) = 1/2 + 1/\pi\arctan (t)$.
\item The $100 \times \gamma$th percentile is $t_{\gamma}(1) = \tan[\pi (\gamma - 1/2)]$.
\end{enumerate}
\end{itemize}
(a) To find the CDF of $T$, we integrate the pdf as follows:
\begin{equation*}
    \begin{split}
        f_T (t;\nu) &= \int_{-\infty}^t \frac{\Gamma\left(\frac{\nu+1}{2}\right)}{\Gamma\left(\frac{\nu}{2}\right)} \frac{1}{\sqrt{\nu\pi}} \left(1+\frac{t^2}{\nu}\right)^{-(v+1)/2} dt \\
        f_T (t;1) &= \int_{-\infty}^t \frac{\Gamma\left(\frac{2}{2}\right)}{\Gamma\left(\frac{1}{2}\right)} \frac{1}{\sqrt{\pi}} \left(1+\frac{t^2}{1}\right)^{-2/2} dt \\
        &= \int_{-\infty}^t \frac{1}{\sqrt{\pi}} \frac{1}{\sqrt{\pi}} \left(1+t^2\right)^{-1} dt \\
        &= \int_{-\infty}^t \left(\frac{1}{\sqrt{\pi}}\right)^2 \left(\frac{1}{1+t^2}\right) dt \\
        &= \frac{1}{\pi} \int_{-\infty}^t \frac{1}{1+t^2} dt \\
        &= \frac{1}{\pi}\left(\tan^{-1}\Big|_{-\infty}^t\right) \\
        &= \frac{1}{\pi}\left(\tan^{-1}(t) - \tan^{-1}(-\infty)\right) \\
        &= \frac{1}{\pi}\left(\tan^{-1}(t)\right) - \frac{1}{\pi}\left(-\frac{\pi}{2}\right) \\
        &= \frac{1}{2} + \frac{1}{\pi}\arctan (t)
    \end{split}
\end{equation*}
(b) Now that we have the CDF of $T$, we can use that to find the $100 \times \gamma$th percentile to be $t_{\gamma}(1) = \tan[\pi (\gamma - 1/2)]$.  An arbitrary $100 \times \gamma$th percentile is defined by
\[\P\left(T \leq t_{\gamma}(1)\right) = \gamma \Longrightarrow F_T \left(t_{\gamma}(1)\right) = \gamma.\]
We can now substitute in the CDF of $T$ and solve for the arbitrary $t_{\gamma}(1)$ as follows:
\begin{equation*}
    \begin{split}
        \frac{1}{2} + \frac{1}{\pi} \arctan \left(t_{\gamma} (1)\right) &= \gamma \\
        \frac{1}{\pi} \arctan \left(t_{\gamma} (1)\right) &= \gamma - \frac{1}{2} \\
        \arctan \left(t_{\gamma} (1)\right) &= \pi\left(\gamma - \frac{1}{2}\right) \\
        t_{\gamma} (1) &= \tan\left[\pi\left(\gamma - \frac{1}{2}\right)\right]
    \end{split}
\end{equation*}
Thus, the $100 \times \gamma$th percentile to be $t_{\gamma}(1) = \tan[\pi (\gamma - 1/2)]$.


\end{document}
